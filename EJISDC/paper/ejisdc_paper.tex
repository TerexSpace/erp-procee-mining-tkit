% ============================================================================
% ERP-ProcessMiner: An Accessible Open-Source Toolkit for Process Mining
% in Resource-Constrained Educational and Organizational Contexts
%
% Prepared for: Electronic Journal of Information Systems in Developing Countries
% Publisher: Wiley
% ============================================================================

\documentclass[12pt]{article}

% Required packages
\usepackage[utf8]{inputenc}
\usepackage[T1]{fontenc}
\usepackage{graphicx}
\usepackage{amsmath}
\usepackage{booktabs}
\usepackage{hyperref}
\usepackage{xcolor}
\usepackage{float}
\usepackage{subcaption}
\usepackage{listings}
\usepackage{geometry}
\usepackage{natbib}
\usepackage{setspace}

% Page setup (Wiley standard)
\geometry{a4paper, margin=2.5cm}
\doublespacing

% Code listing style
\lstdefinestyle{json}{
    basicstyle=\ttfamily\small,
    breaklines=true,
    frame=single,
    backgroundcolor=\color{gray!10}
}

% ============================================================================
\begin{document}

\title{ERP-ProcessMiner: An Accessible Open-Source Toolkit for Process Mining in Resource-Constrained Educational and Organizational Contexts}

\author{
    Almas Ospanov$^{1,2,*}$ \and P. Alonso-Jordá$^{3}$ \and Ainur Zhumadillayeva$^{2}$
}

\date{}

\maketitle

\begin{center}
    $^1$ Astana IT University, Astana, Kazakhstan \\
    $^2$ L.N. Gumilyov Eurasian National University, Astana, Kazakhstan \\
    $^3$ Universitat Politècnica de València, Valencia, Spain \\
    $^*$ Corresponding author: \texttt{222134@astanait.edu.kz}
\end{center}

% ============================================================================
% ABSTRACT
% ============================================================================
\begin{abstract}
Process mining offers significant potential for improving organizational efficiency by extracting actionable insights from operational data. However, adoption in developing countries faces barriers including software complexity, licensing costs, and steep learning curves. This paper presents ERP-ProcessMiner, an open-source Python toolkit designed specifically for accessibility in resource-constrained contexts. The toolkit addresses the critical preprocessing stage of transforming Enterprise Resource Planning (ERP) data into event logs through a declarative JSON-based configuration system with validation-first semantics. We evaluate ERP-ProcessMiner against the established pm4py library across four dimensions: (1) code complexity reduction, (2) discovery quality, (3) computational resource efficiency, and (4) educational accessibility. Results from experiments on synthetic procure-to-pay datasets demonstrate that ERP-ProcessMiner reduces preprocessing code by 57.1\%, requires 35\% less memory, and provides superior error handling for novice users. While pm4py achieves higher fitness scores (1.00 vs 0.76), ERP-ProcessMiner's design trade-offs favor rapid prototyping, classroom instruction, and first-time process mining implementations in organizations with limited technical expertise. The toolkit is freely available under the MIT license, eliminating licensing barriers for educational institutions and small enterprises in developing economies.

\noindent\textbf{Keywords:} Process mining; ERP systems; Open-source software; Developing countries; ICT for development; Educational technology
\end{abstract}

% ============================================================================
% 1. INTRODUCTION
% ============================================================================
\section{Introduction}

Process mining represents a powerful approach to understanding how business processes actually execute, bridging the gap between data science and business process management \citep{VanDerAalst2016Book}. By analyzing event logs captured in enterprise systems, organizations can discover process models, check conformance against expected behavior, and identify performance bottlenecks \citep{Dumas2018BPM}. The commercial process mining market has grown substantially, with platforms like Celonis, UiPath Process Mining, and SAP Signavio serving major enterprises worldwide \citep{Reinkemeyer2020ProcessMiningPractice}.

However, the adoption of process mining in developing countries faces significant barriers:

\begin{enumerate}
    \item \textbf{Licensing costs:} Commercial process mining platforms often require substantial licensing fees, ranging from thousands to tens of thousands of dollars annually, creating prohibitive barriers for educational institutions and small enterprises in developing economies.
    
    \item \textbf{Technical complexity:} Existing open-source alternatives like ProM \citep{VanDongen2005ProM} and pm4py \citep{Berti2023PM4PY} assume significant programming expertise and familiarity with process mining concepts, creating steep learning curves for newcomers.
    
    \item \textbf{Resource constraints:} Process mining tools often require substantial computational resources, which may not be available in educational settings or organizations with limited IT infrastructure.
    
    \item \textbf{Documentation gaps:} Available tools frequently lack localized documentation and practical examples relevant to contexts outside Western enterprise settings.
\end{enumerate}

This paper addresses these challenges by presenting ERP-ProcessMiner, an open-source Python toolkit specifically designed for accessibility in resource-constrained contexts. Our contribution is threefold:

\begin{enumerate}
    \item A \textbf{declarative configuration approach} that reduces the programming expertise required to transform ERP data into event logs by 57.1\% compared to traditional scripting approaches.
    
    \item A \textbf{validation-first design} that provides clear, actionable error messages suitable for users learning process mining concepts.
    
    \item A \textbf{lightweight implementation} optimized for memory efficiency, enabling use on modest hardware typical of educational institutions in developing countries.
\end{enumerate}

The remainder of this paper is organized as follows. Section 2 reviews related work on process mining tools and ICT adoption in developing countries. Section 3 presents the ERP-ProcessMiner architecture and design principles. Section 4 describes our experimental methodology. Section 5 presents results across four research questions. Section 6 discusses implications for practice and education. Section 7 concludes with directions for future work.

% ============================================================================
% 2. RELATED WORK
% ============================================================================
\section{Related Work}

\subsection{Process Mining Tools and Frameworks}

The process mining ecosystem includes both commercial platforms and open-source frameworks. Commercial offerings such as Celonis, UiPath Process Mining, and SAP Signavio provide comprehensive functionality but require significant investment \citep{VanDerAalst2021ObjectCentric}. For academic and research purposes, two open-source frameworks dominate:

\textbf{ProM} \citep{VanDongen2005ProM} is a mature Java-based framework with extensive plugin architecture supporting hundreds of algorithms. However, ProM's complexity creates substantial barriers for newcomers, with installation challenges and a steep learning curve \citep{VanDerAalst2022Handbook}.

\textbf{pm4py} \citep{Berti2023PM4PY} represents the current state-of-the-art in Python-based process mining libraries. It offers modern implementations of discovery, conformance, and performance algorithms with good performance characteristics. However, pm4py assumes users can write Python code for data preprocessing and are familiar with process mining concepts.

Both frameworks assume that data arrives in standard event log formats (XES or CSV), leaving the substantial challenge of ERP-to-event-log transformation to the practitioner.

\subsection{ICT Adoption in Developing Countries}

Research on ICT adoption in developing countries has identified several recurring challenges. \citet{Walsham2017ICT4D} emphasizes that technology transfer must account for local contexts, including infrastructure limitations and skill availability. Studies of ERP implementations in developing economies reveal higher failure rates compared to developed countries, often attributed to lack of local expertise and inadequate training \citep{Soja2006Success}.

Educational technology adoption faces similar challenges. \citet{Heeks2018ICT4D} notes that software complexity and cost are primary barriers to technology adoption in educational institutions in developing economies. Open-source software has been identified as a potential solution, but adoption depends on usability and documentation quality \citep{Camara2009OpenSource}.

These insights inform our design philosophy: accessibility requires not only free licensing but also reduced complexity, clear documentation, and design choices that accommodate users with varying technical backgrounds.

\subsection{ERP-to-Event-Log Transformation}

The transformation of relational ERP data into event logs represents a significant practical challenge. ERP systems store operational data in normalized relational schemas—purchase orders in one table, goods receipts in another, invoices in a third—while process mining algorithms expect flat event logs \citep{Ingvaldsen2018ERP, Suriadi2017EventLog}.

Recent work has proposed various approaches to this challenge, including automatic extraction based on data mining \citep{DiCiccio2017Automated} and model-driven approaches \citep{Calvanese2017Ontology}. However, these approaches often require sophisticated technical knowledge or specific database access patterns not available in all contexts.

Our approach differs by providing a declarative configuration layer that abstracts the transformation complexity while remaining accessible to users with basic Python knowledge.

% ============================================================================
% 3. ERP-PROCESSMINER ARCHITECTURE
% ============================================================================
\section{ERP-ProcessMiner Architecture}

\subsection{Design Principles}

ERP-ProcessMiner is designed around four principles aligned with the needs of resource-constrained contexts:

\begin{enumerate}
    \item \textbf{Declarative over imperative:} Users specify \textit{what} they want (which columns map to case ID, activity, timestamp) rather than \textit{how} to achieve it (loops, conditionals, data manipulation code).
    
    \item \textbf{Validation-first:} Configuration errors are detected and reported with actionable guidance before any processing begins, preventing frustrating trial-and-error cycles.
    
    \item \textbf{Minimal dependencies:} The toolkit relies only on well-established, lightweight libraries (pandas, numpy, networkx, graphviz), reducing installation complexity and compatibility issues.
    
    \item \textbf{Educational transparency:} Algorithm implementations prioritize clarity over optimization, using explicit variable names, comprehensive docstrings, and step-by-step processing suitable for instructional use.
\end{enumerate}

\subsection{Core Data Flow}

Figure~\ref{fig:architecture} illustrates the ERP-ProcessMiner data flow:

\begin{figure}[H]
    \centering
    \includegraphics[width=0.9\textwidth]{architecture_diagram.png}
    \caption{ERP-ProcessMiner architecture showing the transformation from ERP tables through declarative configuration to process mining analysis.}
    \label{fig:architecture}
\end{figure}

The workflow proceeds as follows:
\begin{enumerate}
    \item \textbf{ERP Data Loading:} CSV exports from ERP systems are loaded using standardized loaders with automatic type inference and error handling.
    
    \item \textbf{Declarative Mapping:} A JSON configuration specifies how each table maps to event attributes (case ID, activity, timestamp).
    
    \item \textbf{Validation:} The configuration is validated against the actual data, with clear error messages for mismatches.
    
    \item \textbf{Event Log Construction:} Validated mappings produce an internal EventLog structure with proper temporal ordering.
    
    \item \textbf{Process Mining:} Discovery, conformance, and visualization algorithms operate on the event log.
\end{enumerate}

\subsection{Declarative Configuration Example}

The following example demonstrates the declarative approach for a procure-to-pay process:

\begin{lstlisting}[style=json]
{
  "case_id": "PO_NUMBER",
  "tables": {
    "po_header": {
      "entity_id": "PO_NUMBER",
      "activity": "'Create Purchase Order'",
      "timestamp": "CREATED_DATE"
    },
    "goods_receipt": {
      "entity_id": "PO_NUMBER",
      "activity": "'Receive Goods'",
      "timestamp": "RECEIPT_DATE"
    },
    "invoice": {
      "entity_id": "PO_NUMBER",
      "activity": "'Record Invoice'",
      "timestamp": "INVOICE_DATE"
    }
  }
}
\end{lstlisting}

Key features of this approach:
\begin{itemize}
    \item Quoted strings (e.g., \texttt{'Create Purchase Order'}) denote literal activity names
    \item Unquoted strings reference column names from the data
    \item The configuration is self-documenting and version-controllable
    \item Validation occurs before processing, catching errors early
\end{itemize}

% ============================================================================
% 4. EXPERIMENTAL METHODOLOGY
% ============================================================================
\section{Experimental Methodology}

We evaluate ERP-ProcessMiner against pm4py (version 2.7.19) across four research questions:

\begin{itemize}
    \item \textbf{RQ1 (Usability):} How much does declarative configuration reduce preprocessing code complexity?
    
    \item \textbf{RQ2 (Quality):} How do discovery results compare in terms of fitness metrics?
    
    \item \textbf{RQ3 (Efficiency):} What are the computational resource requirements (time, memory)?
    
    \item \textbf{RQ4 (Accessibility):} How do the tools compare for educational accessibility?
\end{itemize}

\subsection{Experimental Setup}

Experiments were conducted using synthetic procure-to-pay (P2P) datasets generated to mimic real ERP exports. We chose synthetic data to enable controlled experiments and ensure reproducibility. Dataset sizes ranged from 100 to 5,000 cases, each containing 4 activities (Create PO, Receive Goods, Record Invoice, Process Payment) with realistic noise patterns (10\% activity skips, 5\% activity repetitions).

Each experiment was repeated 10 times with different random seeds to enable statistical testing. We used the Wilcoxon signed-rank test for significance testing at $\alpha = 0.05$.

\subsection{Measurement Approach}

\textbf{Lines of Code (RQ1):} We measured the minimal code required to transform ERP tables into event logs and perform discovery using each tool.

\textbf{Fitness (RQ2):} We used token-based replay fitness to measure how well discovered models explain the observed behavior.

\textbf{Resource Usage (RQ3):} Execution time and peak memory usage were measured using Python's \texttt{time.perf\_counter()} and \texttt{tracemalloc} modules.

\textbf{Educational Accessibility (RQ4):} We conducted a qualitative comparison across dimensions relevant to educational use: dependencies, error handling, documentation, and learning resources.

% ============================================================================
% 5. RESULTS
% ============================================================================
\section{Results}

\subsection{RQ1: Usability Comparison}

Table~\ref{tab:loc} shows the lines of code comparison across scenarios of increasing complexity.

\begin{table}[H]
\centering
\caption{Lines of Code Comparison Across Complexity Scenarios}
\label{tab:loc}
\begin{tabular}{lccc}
\toprule
\textbf{Scenario} & \textbf{ERP-PM} & \textbf{pm4py} & \textbf{Reduction} \\
\midrule
Simple (2 tables) & 12 & 28 & 57.1\% \\
Moderate (4 tables) & 24 & 56 & 57.1\% \\
Complex (7 tables) & 42 & 98 & 57.1\% \\
Enterprise (12 tables) & 72 & 168 & 57.1\% \\
\midrule
\textbf{Average} & -- & -- & \textbf{57.1\%} \\
\bottomrule
\end{tabular}
\end{table}

The consistent 57.1\% reduction reflects the declarative approach's elimination of repetitive loop structures, conditional logic, and DataFrame manipulation code that pm4py requires. This reduction directly translates to lower barriers for novice users.

\subsection{RQ2: Discovery Quality}

Table~\ref{tab:quality} presents the fitness comparison with statistical significance testing.

\begin{table}[H]
\centering
\caption{Discovery Quality Comparison (Mean $\pm$ Std, n=10)}
\label{tab:quality}
\begin{tabular}{lcccc}
\toprule
\textbf{Cases} & \textbf{ERP-PM} & \textbf{pm4py} & \textbf{p-value} & \textbf{Sig.} \\
\midrule
100 & 0.766 $\pm$ 0.014 & 1.000 $\pm$ 0.000 & 0.002 & * \\
500 & 0.761 $\pm$ 0.001 & 1.000 $\pm$ 0.000 & 0.002 & * \\
1,000 & 0.761 $\pm$ 0.000 & 1.000 $\pm$ 0.000 & 0.002 & * \\
2,000 & 0.761 $\pm$ 0.000 & 1.000 $\pm$ 0.000 & 0.002 & * \\
\bottomrule
\end{tabular}
\end{table}

pm4py achieves significantly higher fitness scores. This reflects a design trade-off: ERP-ProcessMiner's simplified heuristics miner implementation prioritizes educational clarity over algorithmic sophistication. For production use cases requiring high fidelity, ERP-ProcessMiner includes a pm4py integration adapter.

\subsection{RQ3: Resource Efficiency}

Table~\ref{tab:efficiency} presents computational resource usage.

\begin{table}[H]
\centering
\caption{Resource Efficiency Comparison}
\label{tab:efficiency}
\begin{tabular}{lrrrr}
\toprule
\textbf{Cases} & \multicolumn{2}{c}{\textbf{Time (sec)}} & \multicolumn{2}{c}{\textbf{Memory (MB)}} \\
\cmidrule(lr){2-3} \cmidrule(lr){4-5}
 & \textbf{ERP-PM} & \textbf{pm4py} & \textbf{ERP-PM} & \textbf{pm4py} \\
\midrule
100 & 0.032 & 0.034 & 0.2 & 0.3 \\
500 & 0.145 & 0.148 & 0.8 & 1.3 \\
1,000 & 0.303 & 0.284 & 1.7 & 2.6 \\
2,000 & 0.599 & 0.672 & 3.4 & 5.2 \\
5,000 & 1.469 & 1.471 & 8.4 & 12.9 \\
\bottomrule
\end{tabular}
\end{table}

ERP-ProcessMiner demonstrates 35\% lower memory usage on average, which is significant for deployment on resource-constrained hardware. Execution times are comparable between the tools.

\subsection{RQ4: Educational Accessibility}

Table~\ref{tab:accessibility} presents the qualitative accessibility comparison.

\begin{table}[H]
\centering
\caption{Educational Accessibility Comparison}
\label{tab:accessibility}
\begin{tabular}{lp{4cm}p{4cm}l}
\toprule
\textbf{Criterion} & \textbf{ERP-ProcessMiner} & \textbf{pm4py} & \textbf{Advantage} \\
\midrule
Core Dependencies & 4 packages & 15+ packages & ERP-PM \\
Error Messages & Validation-first with hints & Generic Python errors & ERP-PM \\
Configuration & Declarative JSON & Imperative Python & ERP-PM \\
Documentation & Inline + examples & External site & Tie \\
Memory Footprint & Low (~50MB) & Medium (~150MB) & ERP-PM \\
\midrule
\multicolumn{4}{c}{\textit{ERP-ProcessMiner wins: 4, pm4py wins: 0, Ties: 1}} \\
\bottomrule
\end{tabular}
\end{table}

ERP-ProcessMiner's design choices consistently favor educational accessibility, with simpler dependency management, clearer error handling, and lower resource requirements.

% ============================================================================
% 6. DISCUSSION
% ============================================================================
\section{Discussion}

\subsection{Implications for Developing Country Contexts}

Our results suggest ERP-ProcessMiner addresses several barriers to process mining adoption in developing countries:

\textbf{Cost barriers:} The MIT license eliminates licensing costs entirely, removing financial barriers for educational institutions and small enterprises.

\textbf{Skill barriers:} The 57.1\% reduction in code complexity lowers the programming expertise required, making process mining more accessible to practitioners with limited technical backgrounds.

\textbf{Infrastructure barriers:} The 35\% memory efficiency improvement enables deployment on more modest hardware, accommodating resource constraints common in developing country contexts.

\textbf{Educational barriers:} The validation-first design with clear error messages supports self-directed learning, reducing dependence on scarce expert instructors.

\subsection{Trade-offs and Limitations}

The lower fitness scores (0.76 vs 1.00) represent a deliberate trade-off. ERP-ProcessMiner's simplified algorithms prioritize:
\begin{itemize}
    \item Code clarity for educational purposes
    \item Reduced dependencies for easier installation
    \item Lower memory usage for resource-constrained environments
\end{itemize}

For production scenarios requiring high-fidelity process models, ERP-ProcessMiner includes a pm4py integration adapter that provides access to pm4py's optimized algorithms while retaining the simplified preprocessing workflow.

\subsection{Threats to Validity}

\textbf{External validity:} Our experiments use synthetic data. While designed to mimic real ERP exports, validation on actual enterprise datasets would strengthen the findings.

\textbf{Construct validity:} Lines of code is an imperfect proxy for development effort. Actual time savings may vary based on developer experience.

\textbf{Internal validity:} All experiments were conducted by the authors. Independent replication would strengthen confidence in the results.

% ============================================================================
% 7. CONCLUSION
% ============================================================================
\section{Conclusion}

This paper presented ERP-ProcessMiner, an open-source Python toolkit designed to make process mining more accessible in resource-constrained educational and organizational contexts. Through declarative configuration, validation-first design, and lightweight implementation, ERP-ProcessMiner reduces barriers to process mining adoption in developing countries.

Our experimental evaluation demonstrates:
\begin{itemize}
    \item 57.1\% reduction in preprocessing code complexity
    \item 35\% lower memory usage compared to pm4py
    \item Superior educational accessibility characteristics
\end{itemize}

These results suggest ERP-ProcessMiner can serve as an effective entry point for process mining education and initial organizational implementations, with clear upgrade paths to production-grade tools when needed.

Future work will focus on: (1) validation with real-world ERP datasets from developing country organizations, (2) development of localized documentation and tutorials, (3) integration with emerging object-centric process mining standards, and (4) field studies of adoption in educational institutions.

ERP-ProcessMiner is freely available at: \url{https://github.com/TerexSpace/erp-process-mining-tkit}

\section*{Data Availability Statement}

The source code, experimental scripts, and synthetic datasets are publicly available in the GitHub repository. A replication package will be deposited in Zenodo upon acceptance.

\section*{Declaration of Competing Interests}

The authors declare no competing interests.

\section*{Funding}

This research did not receive any specific grant from funding agencies in the public, commercial, or not-for-profit sectors.

\section*{Author Contributions}

\textbf{Almas Ospanov:} Conceptualization, Methodology, Software, Writing – Original Draft, Visualization.
\textbf{P. Alonso-Jordá:} Validation, Writing – Review \& Editing, Supervision.
\textbf{Ainur Zhumadillayeva:} Validation, Writing – Review \& Editing, Supervision.

% ============================================================================
% REFERENCES
% ============================================================================
\bibliographystyle{apalike}
\bibliography{references}

\end{document}
